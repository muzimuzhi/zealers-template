%================ preamble ===============

%================ package ================	
\usepackage{amsmath, amssymb, amsthm}
\usepackage{bbding}	% \FiveStar
\usepackage[shortlabels,inline]{enumitem}
\usepackage{geometry}
\usepackage{graphics}
%\usepackage{ntheorem}
%\usepackage{pifont}
\usepackage{tabu}
\usepackage[normalem]{ulem} % use normalem to protect \emph
\usepackage{verbatim} % comment environment
\usepackage{xcolor}

\usepackage[colorlinks=true,CJKbookmarks=true]{hyperref}
	% toc's hyperlink is also made by hyperref


%============= theorem style =============
\newtheoremstyle{Question}
  {\topsep}    % ABOVESPACE
  {\topsep} % BELOWSPACE
  {\itshape}   % BODYFONT
  {0pt}        % INDENT (empty value is the same as 0pt)
  {\bfseries}  % HEADFONT
  {:}         % HEADPUNCT
  {0pt plus 1pt minus 1pt} % HEADSPACE
  {}           % CUSTOM-HEAD-SPEC
  
\begin{comment}
\newtheoremstyle{Answer}
  {0.25\topsep}   % ABOVESPACE
  {\topsep}   % BELOWSPACE
  {\normalfont}  % BODYFONT
  {0pt}       % INDENT (empty value is the same as 0pt)
  {\bfseries} % HEADFONT
  {:}         % HEADPUNCT
  {0pt plus 1pt minus 1pt} % HEADSPACE
  {}          % CUSTOM-HEAD-SPEC
\end{comment}
  
\newtheoremstyle{postQuestion}
  {0.5\topsep}   % ABOVESPACE
  {\topsep}   % BELOWSPACE
  {\itshape}  % BODYFONT
  {2em}       % INDENT (empty value is the same as 0pt)
  {\bfseries} % HEADFONT
  {:}         % HEADPUNCT
  {0pt plus 1pt minus 1pt} % HEADSPACE
  {}          % CUSTOM-HEAD-SPEC

\begin{comment}
\newtheoremstyle{postAnswer}
  {0.25\topsep}   % ABOVESPACE
  {\topsep}   % BELOWSPACE
  {\normalfont}  % BODYFONT
  {2em}       % INDENT (empty value is the same as 0pt)
  {\bfseries} % HEADFONT
  {:}         % HEADPUNCT
  {0pt plus 1pt minus 1pt} % HEADSPACE
  {}          % CUSTOM-HEAD-SPEC
\end{comment}

\theoremstyle{Question}
\newtheorem{question}{\color{magenta}问题}

%\theoremstyle{Answer}
%\newtheorem*{answer}{\color{blue}回答}

\theoremstyle{postQuestion}
\newtheorem*{postquestion}{\color{magenta!50}追问}

%\theoremstyle{postAnswer}
%\newtheorem*{postanswer}{\color{blue!50}回答}

\newcommand{\answer}{\par\noindent\textbf{\textcolor{blue}{回答:}}\normalfont\songti}
\newcommand{\postanswer}{\par\textbf{\textcolor{blue!50}{回答:}}\rmfamily\songti}

%============ new theorems ==============
\newcommand\tips{\bgroup\markoverwith
  {\textcolor{yellow}{\rule[-1.05ex]{2pt}{3.55ex}}}\ULon}
	% enable background colored (in yellow) multiline text
	%\rule[-0.5ex]{2pt}{2.5ex}
	% ref: http://tex.stackexchange.com/questions/48501/soul-broken-highlighting-with-xcolor-when-using-selectcolormodel/48549#48549
	% http://bbs.ctex.org/forum.php?mod=viewthread&tid=73680

\newcommand\host[1][主持人]{\textcolor{red}{\bf#1}\par}
\newcommand\lecturer[1][主讲人]{\textcolor{blue}{\bf#1}\par}

\newcommand{\info}[3]{\begin{table}[h]
  \centering
  \begin{tabu}{>{\heiti}X[2,r]>{\kaishu}X[3]}
  	主持人: & #1  \\
  	主讲人: & #2  \\
  	整理者: & #3
  \end{tabu}
  %\label{}
\end{table}
}
