\documentclass[utf8,a4paper]{ctexart}
\title{\bf 20140913 ZJU-Zealers 英语之夜\footnote{本文内容选取自聊天记录,并经志愿者按照自己理解进行汇总。如有错误或不规范,还望指出交流指正。 }}	% \heiti doesn't work for English chars


\usepackage[]{ulem}	% \uline
\usepackage{bbding}	% \FiveStar
\usepackage{pifont}
\usepackage{graphics}
\usepackage[]{color}
\usepackage[]{hyperref}	% toc's hyperlink is also made by hyperref
\usepackage{ntheorem}
{
  \theoremstyle{nonumberplain}
  \theoremheaderfont{\bfseries}
  \theorembodyfont{\mdseries}
  \newtheorem{answer}{\color{blue}答}
}
\usepackage{ntheorem}
{
  \theoremstyle{nonumberplain}
  \theoremheaderfont{\bfseries}
  \theorembodyfont{\mdseries}
  \newtheorem{postquestion}{\color{magenta}追问}
}

\newtheorem{question}{\color{magenta}问题}

% FIXME the \tips can't work properly when the text is multiline
\newcommand\tips[1]{\colorbox{yellow}{#1}}

\newcommand\host[1][主持人]{\textcolor{red}{\bf#1}}
\newcommand\lecturer[1][主讲人]{\textcolor{blue}{\bf#1}}

% \newenvironment{answer}[1]%
% {\textbf{\color{blue}答(#1) }\tt }{}
%       % \tt Fangsong
% it doesn't work

\begin{document}
\maketitle

% \begin{center}
%   \zihao{4} \bf 活动主题\footnote{本文内容选取自聊天记录,并经志愿者按照自己理解进行汇总。如有错误或不规范,还望指出交流指正。 }
% \end{center}

\begin{table}[h]
  \centering
  \begin{tabular}[]{lll}
    \heiti 时间:\kaishu 20140913晚	&	\heiti	主持人: 	&	\kaishu 材料 1-小松鼠\\
    \heiti 地点:\kaishu QQ群		&	\heiti	主讲人: 	&	\kaishu 英语 3-简\\
    \heiti 内容:\kaishu 英语	&	\heiti	整理志愿者: 	&	\kaishu 农资 3-xyz、夕鹰(后期)\\
  \end{tabular}
  \label{tab:basic-information}
\end{table}

\host \qquad \lecturer \qquad \tips{TIPS}
% \textcolor{red}{主持人}
% \textcolor{blue}{主讲人}
% % \textcolor{magenta}{提问与交流}
% \colorbox{yellow}{TIPS}


\tableofcontents

\section{开场白}

\host

Hi,Zealers,我是群主小松鼠,在此特向各位预告晚上的活动~晚上九点,由管理员简给大 家带来英专小科普,由管理员乔及其同专业的小伙伴 concrete 带领大家认识土木行业。 

\textbf{tips}:
\begin{enumerate}
  \item 
Zealers 将于 20:50 停止\textbf{禁言},\textbf{前十分钟}欢迎各位前来\textbf{撒欢}
\item

\textbf{讨论开始后,大家尽量提贴近生活的问题,专业问题请在水群提出(\FiveStarOutline)}。
\item

提问环节,\textbf{主讲已经开始回答问题后禁止提问(\FiveStar)},主讲表示问题回答完毕各位再继续。

\item

昨天和今天是群主和管理员以及同专业小伙伴身先士卒,后期我们将根据这两次试水制定合理的规则。\textbf{请大家随时关注群公告和群投票(\FiveStar)}。(一定养成这个习惯,提高通知效率)
\end{enumerate}




\section{主题演讲}
\lecturer

昨天松鼠和小缪子讲了园艺和材料,文科的童鞋可能有点吃不消。嗯。不要担心...这会通俗易懂的来了。我是 zju 外院英语本科四年级 群里有许多硕博的学长学姐们,班门弄斧 还请多多包涵,哈哈。

外语学院是浙大有名的疗养院,风(nv)景(sheng)秀(zhong)丽(duo)。跟大多数 理工科专业比起来,确实算疗养院了。毕竟我们是不需要做实验做到凌晨的。个人来讲,在英语专业学东西觉得很开心。但是....我是\textbf{绝对不认为学语言轻松的,即便是语言天才,他 也是需要长年累月的积累才能真正发挥自己的语言天赋。我一直很欣赏我们年级第一名的 女生,她对待语言学习严肃认真,}才得以成为至少这个制度下最优秀的人。

\subsection{背单词}
我曾经被无数次问到:\textcolor{red}{你是英语专业的,你们怎么背单词啊?}答案是:大一到大四,我 从来没有“背”过单词...唯一一本单词书是别专业的看我是英语专业送给我的。英专人如果 不是要突击 GRE 等单词量要求高的考试,基本不专门背单词。\textbf{因为最深刻的单词记忆来自 于阅读和视听。}特别是阅读。最近准备专业八级的考试,\tips{有用扇贝记单词,但坚持看 New York times 和一本写美}
\tips{国文化的书,}收获最多。

\subsection{口语 VS 英语能力}
\textcolor{red}{有很多人以为口语好的人,英语一定牛逼,口语不好的人,英语肯定不行。}前半句我没
有实例否认,但是后半句,我必须反对。\textbf{英语专业不一定就口语好},这不是为有些口语不好 的找台阶下,就像学汉语的不一定就中文表达无比流畅一样。我有一个学妹,口语不优秀, 但是她的翻译能力超群,被教授推荐去为 Reader’s Digest 翻译。
\subsection{独特性}
记得系主任讲过,\textbf{英语专业的人,要相信自己的独特性}——的确,别的专业的都会讲 英语,有的还讲得比你英语专业的好,但这并不代表英语专业没有竞争力。所以\textcolor{red}{英语专业跟 英语讲得好的别的专业的人,有什么区别?}下午回去看了下培养方案,除了学校的例行课程, \textbf{英语专业本科课程设计是}:

\begin{enumerate}
    \bfseries
  \item 
基础课程:视听说写+翻译基础
\item
导论性质课程:文学导论+社会学研究方法+文化研究基础+西方思想入门
\item
模块课程:文学、语言学、文化、经贸、口译、笔译
\end{enumerate}



而很多时候,对英语专业能力的鉴别,仅仅停留在了基础层面——听,说,读,写。尤
其对口语的崇尚最盛。即便英专学生也难逃窠臼。\uline{模块课程,基本作用就是提供给英专的学 生们了解各个领域,选择自己以后的研究或者就业方向的机会。}问:\textcolor{red}{学英语搞什么研究?}由 于浙大不是专门的外语学校,所以我截取了\textbf{上海外国语大学英语语言文学方向的研究生方 向细分}。这仅仅是英语文学,还没有算上口译方向。大家请看

浏览完我的专业领域,不得不承认,我在语言学和文学上是不折不扣的渣渣...个人比较 喜欢哲学,文化一类的东西,结合生活想来想去特别好玩特别开心。

\textcolor{red}{英专跟口语好的其他专业的还有一个区别在于},英专对英文写作的要求可以说到了吹毛 求疵的境地,不论是学术论文还是普通议论文还是考试文。特别感谢学院的写作老师四年如 一日呕心沥血的教导——\textbf{格式、用词、用句,甚至一个小小的标点,都逃不过他们的法眼。} 所以每次有小伙伴让我帮忙看英语作文,看完几乎只能苦笑——这作文拿到我们老师手下绝 对返工.......
\subsection{学习方法}
怎么学英语呢?~~ 我的答案是:努力学。这句废话的言外之意是你\textbf{速成不得~听说读写四样,构成基础。
特别是阅读,读得越多,体会越深~ }

\textbf{听:个人求精不求多。}
\tips{只用可可英语,听写 BBC VOA。偶尔听听 Ted}
\tips{和网易公开课怡情看一部美剧很多遍——吸血鬼日记快被我看烂了,最后基}
\tips{本可以不用字幕}——只不过有时候
Damon 讲话太快,sigh...

\textbf{写}:\uline{英专大部分的课都是写论文},还要考专四专八,还有\uline{写作课},所以写作并没有刻意去提高。\tips{想提高写作能力的童鞋推荐选外院的<交际英语写作>}
\tips{课,或者准备准备托福考试}。

\textbf{读}:\tips{国外的杂志 new york times, guardian, economist 以及【与自己阅读}
\tips{能力相适或者稍高一点的】小说——没有之前的积累或者较高的语言能力直接}
\tips{去读 Virginia Woolf, Charles Dickens,}
真的很头疼

差不多讲到这里,欢迎大家的问题~


\section{提问与交流}
\begin{question}[力学 2-尘缘 21:15:34]
  想考个托福,怎么准备
\end{question}
\begin{answer}
  托福考试准备 咳咳。我有新东方的笔记可以分享给大家
\end{answer}

\begin{question}[土木 3-乔 21:19:26]
  如果想快速的提高英语,该怎么破?
\end{question}

\begin{answer}
  T T 基本没办法速成,参照我最后的几条,天天练,其实时间不长的
\end{answer}

\begin{postquestion}[临床医学-2-兔子]
  怎么练?
\end{postquestion}

\begin{answer}
  零碎时间
\end{answer}

\begin{question}[电路与系统 1-郁 21:21:43]
  提高词汇量有什么好方法?感觉背单词好单调枯燥。。。
\end{question}
\begin{answer}
  建议阅读+听力,单词量自然 up
\end{answer}

\begin{postquestion}[力学 2-尘缘 21:22:19]
  阅读不好
\end{postquestion}

\begin{answer}
  选择和自己的程度相适应的,可以从书虫系列开始看
\end{answer}

\begin{question}[应数 1-ideal 21:22:45]
  听力一天练多久
\end{question}

\begin{answer}
  \tips{听力我一天听可可英语一个 BBC}
\end{answer}

\begin{question}[动科 2-蜗牛 21:22:50]
  看老友记,背台词,有用么?只想提高英语口语
\end{question}
\begin{answer}
  口语也要靠词汇量的,你知道和你会用的词汇不是一个数量级
\end{answer}

\begin{question}[动科 3-蛋白质 21:23:43]
  看美剧能提升英语实际会话能力嘛?
\end{question}

\begin{answer}
  看美剧边写笔记可以学习一些常用表达,关键是要做笔记。。。
\end{answer}

\begin{question}[动科 3-蛋白质 21:25:28]
  做笔记的话有没有什么推荐的方法?
\end{question}

\begin{answer}
  记就是记下里面的比较好的短语啊~就遇到自己喜欢的短语 句子 典故 停下来摘抄
\end{answer}

\begin{question}[动科 3-蛋白质 21:26:50]
  加入一个字幕组干干翻译之类的话应该就算一种很速成的方 法了吧
\end{question}
\begin{answer}
  加字幕组干翻译确实有用嗯 我也有同学在做,不过这个时间要求多
\end{answer}

\begin{question}[电路与系统 1-郁 21:29:43]
  简对英文文学有了解吗?英文诗歌跟汉语的古诗词一样有 类似平仄韵律的要求吗?感觉英文怎么也表达不出像汉语关关雎鸠, 在河之洲这样的意境 和美感。
\end{question}

\begin{answer}
  T 有些英文诗歌有韵律的,英语文体学会讲到 英语押头韵 押尾韵 还有别的 考虑到辅 音原音的组合
\end{answer}

\begin{question}[应数 1-ideal 21:31:25]
  一般查不认识的单词有道靠谱不?
\end{question}

\begin{answer}
  有道不靠谱 推荐\tips{爱词霸}
\end{answer}


\begin{question}[电路与系统 1-郁 21:32:54]
  英文诗也是有古诗和现代诗之说?
\end{question}

\begin{answer}
  有的 有中世纪诗歌 最老的就是贝尔武夫 后来英语文学还分了很多时期 比如浪漫主义 时期 新古典主义时期
\end{answer}

\begin{question}[园艺 2-小缪子 21:37:14]
  口语很烂 都羞于开口啊
\end{question}
\begin{answer}
  一定要开口 ~大家都是一步一步走过来的~
\end{answer}

\begin{question}[历史 2-阿呜 21:39:15]
  玉泉和西溪貌似都没有英语角,只有紫金港有
\end{question}

\begin{answer}
  玉泉北门出去有几个咖啡厅 貌似有 language exchange 的活动
\end{answer}

\begin{question}[药学 3-牛牛 21:48:04]
  那怎么能用鼻腔呢?
\end{question}

\begin{answer}
  这个嘛 鼻腔这块 我靠模仿
\end{answer}

\begin{question}[药学 3-牛牛 21:51:49]
  是任何美剧都可以练习口语听力吗?
\end{question}
\begin{answer}
  \tips{Nope 有些不合适 太快或者是太 professional 老师们经常推荐 friends 配}
\tips{合听 VOA 或者BBC}
\end{answer}

\begin{question}[园艺 2-小缪子 21:37:14]
  口语很烂 都羞于开口啊
\end{question}
\begin{answer}
  一定要开口 ~大家都是一步一步走过来的~
\end{answer}

\section{志愿者有话说}
夕鹰:一切尽在不言中
% This is a $\mistake$.
% And this is $\another$

\end{document}
