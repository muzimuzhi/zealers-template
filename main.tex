%===== please compile with XeLaTeX =====
\documentclass[a4paper,12pt]{ctexart}
\title{\textbf{ZJU-Zealers 英语之夜}%
	\footnote{本文档内容选自QQ群聊记录,汇总时难免夹入整理者个人理解。如遇错误或不规范,请不吝指正。}
	\footnote{作为公开模板,本文档只作格式展示,不呈现具体内容。}
}% \heiti doesn't work for English chars
\author{}
\date{2014年11月23日}


%================ preamble ===============

%================ package ================	
\usepackage{amsmath, amssymb, amsthm}
\usepackage{bbding}	% \FiveStar
\usepackage[shortlabels,inline]{enumitem}
\usepackage{geometry}
\usepackage{graphics}
%\usepackage{ntheorem}
%\usepackage{pifont}
\usepackage{tabu}
\usepackage[normalem]{ulem} % use normalem to protect \emph
\usepackage{verbatim} % comment environment
\usepackage{xcolor}

\usepackage[colorlinks=true,CJKbookmarks=true]{hyperref}
	% toc's hyperlink is also made by hyperref


%============= theorem style =============
\newtheoremstyle{Question}
  {\topsep}    % ABOVESPACE
  {\topsep} % BELOWSPACE
  {\itshape}   % BODYFONT
  {0pt}        % INDENT (empty value is the same as 0pt)
  {\bfseries}  % HEADFONT
  {:}         % HEADPUNCT
  {0pt plus 1pt minus 1pt} % HEADSPACE
  {}           % CUSTOM-HEAD-SPEC
  
\begin{comment}
\newtheoremstyle{Answer}
  {0.25\topsep}   % ABOVESPACE
  {\topsep}   % BELOWSPACE
  {\normalfont}  % BODYFONT
  {0pt}       % INDENT (empty value is the same as 0pt)
  {\bfseries} % HEADFONT
  {:}         % HEADPUNCT
  {0pt plus 1pt minus 1pt} % HEADSPACE
  {}          % CUSTOM-HEAD-SPEC
\end{comment}
  
\newtheoremstyle{postQuestion}
  {0.5\topsep}   % ABOVESPACE
  {\topsep}   % BELOWSPACE
  {\itshape}  % BODYFONT
  {2em}       % INDENT (empty value is the same as 0pt)
  {\bfseries} % HEADFONT
  {:}         % HEADPUNCT
  {0pt plus 1pt minus 1pt} % HEADSPACE
  {}          % CUSTOM-HEAD-SPEC

\begin{comment}
\newtheoremstyle{postAnswer}
  {0.25\topsep}   % ABOVESPACE
  {\topsep}   % BELOWSPACE
  {\normalfont}  % BODYFONT
  {2em}       % INDENT (empty value is the same as 0pt)
  {\bfseries} % HEADFONT
  {:}         % HEADPUNCT
  {0pt plus 1pt minus 1pt} % HEADSPACE
  {}          % CUSTOM-HEAD-SPEC
\end{comment}

\theoremstyle{Question}
\newtheorem{question}{\color{magenta}问题}

%\theoremstyle{Answer}
%\newtheorem*{answer}{\color{blue}回答}

\theoremstyle{postQuestion}
\newtheorem*{postquestion}{\color{magenta!50}追问}

%\theoremstyle{postAnswer}
%\newtheorem*{postanswer}{\color{blue!50}回答}

\newcommand{\answer}{\par\noindent\textbf{\textcolor{blue}{回答:}}\normalfont\songti}
\newcommand{\postanswer}{\par\textbf{\textcolor{blue!50}{回答:}}\rmfamily\songti}

%============ new theorems ==============
\newcommand\tips{\bgroup\markoverwith
  {\textcolor{yellow}{\rule[-1.05ex]{2pt}{3.55ex}}}\ULon}
	% enable background colored (in yellow) multiline text
	%\rule[-0.5ex]{2pt}{2.5ex}
	% ref: http://tex.stackexchange.com/questions/48501/soul-broken-highlighting-with-xcolor-when-using-selectcolormodel/48549#48549
	% http://bbs.ctex.org/forum.php?mod=viewthread&tid=73680

\newcommand\host[1][主持人]{\textcolor{red}{\bf#1}\par}
\newcommand\lecturer[1][主讲人]{\textcolor{blue}{\bf#1}\par}

\newcommand{\info}[3]{\begin{table}[h]
  \centering
  \begin{tabu}{>{\heiti}X[2,r]>{\kaishu}X[3]}
  	主持人: & #1  \\
  	主讲人: & #2  \\
  	整理者: & #3
  \end{tabu}
  %\label{}
\end{table}
}



%============== document ===============
%=======================================
\begin{document}
\maketitle
\info{材料 1-小松鼠} % host
	 {英语 3-简} % lecturer(s)
	 {农资 3-xyz、夕鹰(后期)} % volunteer(s)
	 
% example
\begin{center}
	\textcolor{red}{主持人} \quad \textcolor{blue}{主讲人} \quad \colorbox{yellow}{Tips}
\end{center}

\tableofcontents

\section{开场白}
\host
Hi, Zealers,我是群主小松鼠,在此特向各位预告晚上的活动$\!\sim$ 晚上九点,由管理员简给大家带来英专小科普,
由管理员乔及其同专业的小伙伴 concrete 带领大家认识土木行业。 

\textbf{Tips}:
\begin{enumerate}
	\item Zealers 将于 20:50 停止\textbf{禁言},\textbf{前十分钟}欢迎各位前来\textbf{撒欢}。
	\item \textbf{讨论开始后,大家尽量提贴近生活的问题,专业问题请在水群提出}(\FiveStar)。
	\item 提问环节,\textbf{主讲已经开始回答问题后禁止提问}(\FiveStar),主讲表示问题回答完毕各位再继续。
	\item 昨天和今天是群主和管理员以及同专业小伙伴身先士卒,后期我们将根据这两次试水制定合理的规则。
	  \textbf{请大家随时关注群公告和群投票}(\FiveStar)(一定养成这个习惯,提高通知效率)。
	  \end{enumerate}

\section{主题演讲}
\lecturer

昨天松鼠和小缪子讲了园艺和材料,文科的童鞋可能有点吃不消。嗯。不要担心 \ldots\ 这会通俗易懂的来了。我是……\,,班门弄斧 还请多多包涵,哈哈。


\subsection{背单词}
我曾经被无数次问到:\textcolor{red}{你是英语专业的,你们怎么背单词啊?}答案是:大一到大四,我从来没有“背”过单词……唯一一本单词书是别专业的看我是英语专业送给我的。

英专如果不是要突击 GRE 等单词量要求高的考试,基本不专门背单词。因为最深刻的单词记忆来自于阅读和视听。

\subsection{口语 v.s.英语能力}
文本强调的一些例子:
\begin{table}[h]
	\centering
	\begin{tabu*}{lll}
		效果名 & 输入 & 输出 \\ \hline
		加粗 & \verb+\textbf{粗粗的}+ & \textbf{粗粗的} \\
		下划线 & \verb+\uline{一条线}+ & \uline{一条线} \\
		文字颜色 & \verb+\textcolor{red}{红红的}+ & \textcolor{red}{红红的} \\
		高亮 & \verb+\tips{蓝蓝的}+ & \tips{蓝蓝的}
	\end{tabu*}
\end{table}

\subsection{独特性}
记得系主任讲过,\textbf{英语专业的人,要相信自己的独特性}——的确,别的专业的都会讲 英语,有的还讲得比你英语专业的好,但这并不代表英语专业没有竞争力。所以\textcolor{red}{英语专业跟 英语讲得好的别的专业的人,有什么区别?}下午回去看了下培养方案,除了学校的例行课程,\textbf{英语专业本科课程设计是}:

\begin{enumerate}
	\bfseries
	\item 基础课程:视听说写+翻译基础
	\item 导论性质课程:文学导论+社会学研究方法+文化研究基础+西方思想入门
	\item 模块课程:文学、语言学、文化、经贸、口译、笔译
\end{enumerate}

\subsection{学习方法}
……

差不多讲到这里,欢迎大家的问题~


\section{提问与交流}
\begin{question}[力学 2-尘缘 21:15:34]
  想考个托福,怎么准备
  
  \answer 托福考试准备\ 咳咳。我有新东方的笔记可以分享给大家
\end{question}

\begin{question}[土木 3-乔 21:19:26]
  如果想快速的提高英语,该怎么破?
  
  \answer T T 基本没办法速成,参照我最后的几条,天天练,其实时间不长的。
\end{question}

\begin{postquestion}[临床医学-2-兔子]
  怎么练?
  
  \postanswer 零碎时间
\end{postquestion}

\begin{question}[电路与系统 1-郁 21:21:43]
  提高词汇量有什么好方法?感觉背单词好单调枯燥。{}。{}。
  
  \answer 建议阅读+听力,单词量自然 up
\end{question}

\section{整理者的自留地}
夕鹰:一切尽在不言中

\bigskip
\noindent 慕子

遗留问题:能自动断行的中文文本高亮,既有解决方案不够完美。

可改进的方面:
\begin{enumerate*}[i)]
	\item 配色,
	%\item 内容与格式的分离,
	%\item 作为整理稿,行文尽量加标点、标点全部中文化,
	\item 补充图表插入的简明例子。
	%\item (个人意见)降低着重记号(现有加粗、下划线、标红、高亮四种)的种类,并控制使用频次 
\end{enumerate*}
\end{document}
