%===== please compile with XeLaTeX =======
\documentclass[a4paper,12pt]{ctexart}
\title{\textbf{ZJU-zealers 学科之夜}%
	\footnote{本文档整理自QQ群聊记录,或夹入了整理者的个人理解。如遇错误或不规范之处,请不吝指正。}
	\footnote{作为公开模板,本文档只作格式展示,不呈现具体内容。}
}
\author{}
\date{2014年mm月dd日}


%================ preamble ===============
%=========================================

%================ package ================	
\usepackage{amsmath, amssymb, amsthm}
\usepackage[shortlabels,inline]{enumitem}
\usepackage{geometry}
\usepackage{graphics}
\usepackage{tabu}
\usepackage[normalem]{ulem} % use normalem to protect \emph
\usepackage{xcolor}

\usepackage[colorlinks=true,CJKbookmarks=true]{hyperref}


%============= theorem style =============
\newtheoremstyle{Question}
  {\topsep}    % ABOVESPACE
  {\topsep} % BELOWSPACE
  {\itshape}   % BODYFONT
  {0pt}        % INDENT (empty value is the same as 0pt)
  {\bfseries}  % HEADFONT
  {:}         % HEADPUNCT
  {0pt plus 1pt minus 1pt} % HEADSPACE
  {}           % CUSTOM-HEAD-SPEC
  
\newtheoremstyle{postQuestion}
  {0.5\topsep}   % ABOVESPACE
  {\topsep}   % BELOWSPACE
  {\itshape}  % BODYFONT
  {2em}       % INDENT (empty value is the same as 0pt)
  {\bfseries} % HEADFONT
  {:}         % HEADPUNCT
  {0pt plus 1pt minus 1pt} % HEADSPACE
  {}          % CUSTOM-HEAD-SPEC

\theoremstyle{Question}
\newtheorem{question}{\color{magenta}问题}

\theoremstyle{postQuestion}
\newtheorem*{postquestion}{\color{magenta!50}追问}

%============ new theorems ==============
\newcommand{\answer}{\par\noindent\textbf{\textcolor{blue}{回答:}}\normalfont\songti}
\newcommand{\postanswer}{\par\textbf{\textcolor{blue!50}{回答:}}\rmfamily\songti}

\newcommand\tips{\bgroup\markoverwith
  {\textcolor{yellow}{\rule[-1.05ex]{2pt}{3.55ex}}}\ULon}
	% enable background colored (in yellow) multiline text
	%\rule[-0.5ex]{2pt}{2.5ex}

\newcommand\host[1][主持人]{\textcolor{red}{\bf#1}\par}
\newcommand\lecturer[1][主讲人]{\textcolor{blue}{\bf#1}\par}

\newcommand{\info}[3]{\begin{table}[h]
  \centering
  \begin{tabu}{>{\heiti}X[2,r]>{\kaishu}X[3]}
  	主持人: & #1  \\
  	主讲人: & #2  \\
  	整理者: & #3
  \end{tabu}
  %\label{}
\end{table}
}


%============== document ===============
%=======================================
\begin{document}
\maketitle
\info{主持人} % host
	 {主讲人} % lecturer(s)
	 {整理者} % volunteer(s)
	 
% example
\begin{center}
	\textcolor{red}{主持人} \quad \textcolor{blue}{主讲人} \quad \colorbox{yellow}{Tips}
\end{center}

\tableofcontents

\section{开场白}
\host
Hi

\section{主体介绍}
\lecturer
Hi

\subsection{第一部分}
第一点……

第二点……

\subsection{第二部分}
……


\section{提问与交流}
\begin{question}[提问者]
  问题内容
  
  \answer  回答内容
\end{question}

\begin{question}[提问者]
  问题内容
  
  \answer  回答内容
\end{question}

\begin{postquestion}[追问者]
  追问内容
  
  \postanswer 回答内容
\end{postquestion}

\begin{question}[提问者]
  问题内容
  
  \answer  回答内容
\end{question}

\section{未尽事项}
文本强调的一些例子:
\begin{table}[h]
	\centering
	\begin{tabu*}{lll}
		效果名 & 输入 & 输出 \\ \hline
		加粗 & \verb+\textbf{粗粗的}+ & \textbf{粗粗的} \\
		下划线 & \verb+\uline{一条线}+ & \uline{一条线} \\
		文字颜色 & \verb+\textcolor{red}{红红的}+ & \textcolor{red}{红红的} \\
		高亮 & \verb+\tips{蓝蓝的}+ & \tips{蓝蓝的}
	\end{tabu*}
\end{table}

\end{document}
